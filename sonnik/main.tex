\documentclass[openany, oneside]{book}

\input{preamble}
\input{Russian}

\newcommand{\vf}{\vspace{5mm}}

\usepackage{tempora}

\usepackage{epigraph}

\usepackage{geometry}
\geometry{a4paper, top=30mm, left=30mm, right=30mm, bottom=35mm}

\renewcommand{\epigraphsize}{\normalsize}
\renewcommand{\footnotesize}{\normalsize}

% \hyphenpenalty = 10000
% \fontdimen2\font=0.7ex
% \fontdimen3\font=0.4em
\renewcommand{\baselinestretch}{1.015}

\sloppy
\begin{document}

\thispagestyle{empty}

\emptysymbol

\vspace{70mm}

\includegraphics[width=10cm]{pictures/sonnik2.png}

\vspace{10mm}

\ \ \ \ \ \ {\LARGE \bfseries КАЛЕНДАРЬ СНОВИДЕНИЙ}
% \ \ \ \ \ \ {\LARGE \bfseries Жизнь и сновидения — страницы одной и той же книги.}

\vspace{7mm}

\ \ \ \ \ \ {\LARGE Роман Максимович}

\large

\newpage

\emptysymbol

\vspace{2cm}

\begin{flushleft}
{\it

<<Крикнуть что есть сил;

Вдруг мы всё же спим?

\vf

Мимо пусть летят стрелы, слёзы, дни\dots

Всё равно они нас не ранят.

\vf

Придумай мне день,

Где нет ни людей, ни стен, ни ран\dots>>}

\vf

--- Сироткин, <<К водопадам>>

\vf
\vf

\rule{\textwidth}{.7pt}

\vf
\vf

{\it <<При научном исследовании сновидений мы исходим из предположения, что сновидение является результатом нашей душевной деятельности. Тем не менее готовое сновидение является нам чем-то чуждым, в создании чего мы, на наш взгляд, настолько мало повинны, что выражаем это даже в своем языке: «мне снилось». Откуда же проистекает эта «чуждость» сновидения?>>}

\vf

--- Зигмунд Фрейд, <<Толкование сновидений>>

\vf
\vf

\rule{\textwidth}{.7pt}

\vf
\vf

{\it <<Проснись, Ержан!..>>}

\vf

--- Неизвестный

\end{flushleft}

\fontdimen2\font=0.35em
\fontdimen3\font=0.1em

\chapter*{Пролог}
На календаре 2113 год. Люди осуществили космическую экспансию? Может, вошли в контакт с инопланетными существами? Нет, родной Земле ещё долго быть колыбелью Человечества. Тогда, наступил век сверхразвитых технологий? Искусственного интеллекта? Цифрового бессмертия? Нет. Чуть быстрее интернет, чуть дольше работают старые добрые литий-ионные батарейки. У Человечества есть дела поважнее.

Это началось пять лет назад, когда одному норвежскому мальчику приснилось, будто он ест дивного качества бутерброд с тунцом. Не окончив трапезу, мальчик был разбужен родителями в школу. К своему удивлению, лёжа в кровати, он держал остаток бутерброда в руке. С этого момента стало известно, что не только реальность влияет на сны людей, но и наоборот. Инциденты множились: женщина из Италии увидела во сне новую стиральную машину, и к утру была приятно удивлена. Мужчина из Китая пар\'{и}л над кроватью после того, как летал во сне. Интернет пестрел яркими заголовками, но многие попросту не верили. Даже если Иисус Христос самолично предстанет перед заядлым атеистом и пройдётся по воде, тот, скорее всего, сочтёт это обманом зрения, розыгрышем или галлюцинацией. Рационализация~--- страшная сила.

Всё изменилось один год спустя. Один экологический активист, заснув в трущобах Рио де Жанейро, увидел сон, в котором Земля вновь была богата природными ресурсами, корпорации не загрязняли окружающую среду, а бедные страны имели возможность занять достойное место в мировой экономике. Представьте себе удивление парня, когда тот проснулся!

Весь мир изменился в мгновение ока. Вот теперь мировая общественность обратила серьёзное внимание на феномен <<обратного воплощения>>. А что, если во сне можно убить? Устроить стихийное бедствие? Да что там, уничтожить всю планету разом?! Начался хаос на фоне умиротворяющих природных пейзажей. Онейрология, наука о сновидениях, получила чудовищный толчок. Первым делом был изобретён <<онейронный ингибитор>>~--- устройство, позволяющее частично глушить мозговую активность во время сна, предотвращая возникновение сновидений как таковых. Онейронные ингибиторы, или <<нейрины>>, были сравнительно дёшевы в производстве, а своим действием накрывали целую область в радиусе 50 метров. Поэтому, уже к 2110 году эти устройства покрывали все места проживания людей невидимой сетью. Тех, кто жил вне населённых пунктов, стали принудительно селить в города, а сон вне сети ингибиторов стал караться законом во всех странах без исключения. Люди возмущались, но умирать в реальности от чьего-то кошмара тоже не хотели.

Были, однако, и минусы. Действие <<нейринов>> приводило к побочным эффектам: сонливость, лёгкая паранойя, потеря концентрации. Для их устранения начали работу специализированные клиники. Кроме того, прогресс не стоял на месте. С каждым годом выходили всё более совершенные версии ингибиторов, создающие всё меньше побочных эффектов. А для богатых выпускались высокотехнологичные <<оптимизаторы сновидений>>, которые имплантировались в мозг и позволяли своим носителям видеть сны, но только маломасштабные и безопасные. Бонусом шёл документ <<об отсутствии необходимости нахождения под круглосуточным влиянием сети устройств типа ``онейронный ингибитор''>>. Те, кто мог себе это позволить, стали жить в частных домах за городом.

К счастью, масштабных <<обратных воплощений>> после инцидента с бразильским активистом и до покрытия <<нейринами>> не было. Феномен носил преимущественно локальный и незначительный характер. Что же известно науке 2113 года об этом феномене? Всё ещё очень мало. Главным образом, эффект достигается засчёт того, что человек, видящий сон, воспринимает его как объективную реальность. По этой причине осознанные сны никогда не воплощались. Если же человек имеет высокий параметр WP (<<willpower>>~--- <<сила воли>>), сильное эмоциональное напряжение во сне может отразиться на реальности. Стандартные показатели WP варьировались от 200 до 350 пунктов. Если же человек имел показатель ниже 100 пунктов, ему выдавался документ <<об отсутствии необходимости\dots>>. Надёжно предсказать <<обратное воплощение>> пока так и не удалось.

В то время как онейрология развивалась как бешеная, другие области науки и производства сбавили обороты. Всему причина~--- обновление природных ресурсов Земли. Исчезли проблемы загрязнения и глобального потепления, не стало острой нужды в альтернативных источниках энергии и экологичном производстве.

Отдельной головной болью для правительства стали организации сопротивления (так называемые <<мечтатели>> или <<сновидцы>>), которые боролись за ментальную свободу и заявляли, что не позволят государствам иметь тотальный круглосуточный контроль над умами граждан. Однако это уже другая история.\\

Антон Валентинович Страусов~--- старший врач 5-й Санкт-Петербургской государственной онейрологической клиники. Солнечным октябрьским утром (да-да, солнечным) он просыпается от головной боли. На часах 06:00, Вторник, 3 октября. Люди в возрасте старше 50 лет сильнее всех подвержены побочным эффектам <<нейринов>>. Антон выпивает таблетки и идёт на балкон курить. Воздух всё равно чистый, так почему бы и не покурить?

--- Опять голова болит?~--- сзади подходит Настя, его жена.~--- Может, купим тебе всё-таки <<оптимизатор>>? Я слышала, они теперь сигналы <<нейринов>> глушат, можно в городе носить.

--- Ой, да ничего у меня не болит, успокойся\dots \ Лучше пусть Даня в хороший университет поступит.

--- Ну, это ещё если поступит\dots \ Ты же знаешь, он не очень любит учиться\dots

--- Ладно, хватит об этом. Мне уже скоро на работу, я пойду одеваться.\\

Антон едет на машине по утреннему Петербургу. Асфальтовые полосы дорог огорожены деревьями и кустами, на столбах растёт плющ. Незаметные коробочки <<нейринов>> висят на ветках и проводах. Входящий звонок.

--- Здоров, Антоха! Можешь нас с Катькой быстренько до хаты подкинуть?

--- Привет, Никита. Прости, я уже еду на работу. Вызовите сегодня такси.

--- Да ладно тебе, не ломайся! Сюда такси долго идёт. Нельзя, что ли, на 15 минут опоздать? Вы там что, опять свои секретные эксперименты на людях проводите?

--- Господи, Никита, ну какие, к чёрту, эксперименты?! Мы ведь это уже проходили\dots \ Нельзя же верить каждому конспирологическому посту в интернете!

--- Я, Антоха, вообще ничему не верю.

Никита Александрович Высокогорный, лидирующий российский специалист по теоретической онейрологии, вообще ничему и никому не верил. Ему, впрочем, тоже верить было опрометчиво.

--- Короче, нет, и точка. Кто здесь учёный? Правильно, вот и придумай что-нибудь,~--- Антон сбрасывает звонок.\\

Антон поднимается по лестнице ко входу в клинику. Показав пропуск охраннику, он заходит в лифт и нажимает красную кнопку -19 этажа.

\chapter{Август}

\setlength{\epigraphwidth}{.6\textwidth}

\epigraph{\textit{Август~--- это конец лета. Для многих это последний месяц беззаботной жизни. Последняя нота в мелодии. Весело шелестят зелёные кроны деревьев, дует тёплый ветер над головой, солнце смотрит на нас ярким желтком в чистом небе. Но есть что-то печальное в этом шелесте крон, будто ветер в последний раз проносится сквозь зелёные, но уже слегка пожухлые листья. Крупные, спелые яблоки, сочные вишни висят на деревьях, кусты пестрят тысячами зрелых ягод. Флора отдаёт миру свои плоды, своё предсмертное завещание. Понедельник начинается в воскресенье, а сентябрь начинается в августе. Конец? Нет, август~--- это лишь начало.}}

На пляжу я лежу\dots \ Точнее, стою. Ноги утопают в песке, ноги \textit{не мои}. Солнце горит ядовито-зелёным светом над \textit{не моей} головой. По одну сторону блестит бескрайнее голубое море, по другую~--- океан жёлтого песка. Между ними на цветовой палитре лежит ядовито-зелёное небо. А может, небо~--- лишь пустота, иллюзия, и два океана сливаются в один первичный бульон.

<<На какой половине я нахожусь..?>>

Под ногами вода.

<<Смена перспективы. Но что изменилось..?>>

Пустота одинакова с обеих сторон.

Этот пляж не всегда был пустым. Когда-то давно, на нём стояли роскошные отели, росли исполинские пальмы, по песку дети радостно убегали от египтян-аниматоров, пока их родители отлёживали бока на шезлонгах. А в море на десятки метров в глубину росли разноцветные кораллы, между ними оживлённо проплывали морские черепахи, осьминоги, скаты и каракатицы. Чего только не было среди морской природы! Большие и пёстрые раковины, замысловатые рыбьи скелеты прятались в коралловых рифах, а некоторые поговаривали, что на дне лежит человеческий череп. Жемчужиной подводного мира были рыбы: крылатки, антиасы, попугаи, бабочки, наполеоны, солдаты и клоуны мелькали в расщелинах морского рельефа. На берегу же крылатки выступали на передвижных сценах вместе с попугаями, клоуны развлекали весёлую публику, наполеоны всё так же гордо отлёживали бока на шезлонгах, а солдат, на счастье, нигде видно не было.

Сейчас лишь мёртвый песок устилает берег моря. Простая картина: синий, жёлтый, зелёный. И я~--- никто, живая пустота. У меня нет ни тела, ни разума. А если и есть, то они остались там, у широких шезлонгов и исполинских пальм, у коралловых рифов и сонных морских черепах. 

<<Во мне же ничего не осталось, глупый сон и усталость\dots \ Глупый сон и усталость\dots>>\footnote{Сироткин, <<К водопадам>>}

\begin{center}
    * * *
\end{center}

Завтра утром мы улетаем в Петербург. Вещи уже собраны, номер убран. И сейчас, безлюдным вечером, я иду по песчаному пляжу, рядом с тихим Красным морем. Слева шелестят пальмы, горят окна отелей. Справа доносится ровный шум прибоя. Обычно берег завален бутылками из-под пива, окурками и прочим мусором, но сейчас он идеально чист. Видимо, недавно проводили уборку. Небо подёрнуто нежными перистыми облаками, солнце подходит к морскому горизонту. Оно собирается скрыться, но всё медлит, будто издевается надо мной\dots \ Его отражение в воде похоже на белого лебедя, сотканного из света. Лебедь печально улыбается мне, исполняя свой лебединый танец на волнах.

Закат. \textit{Конец.}

Вокруг всё так тихо и спокойно, что аж тошно.

Мы прилетели в Египет в начале августа. Целый месяц на пляже Красного моря~--- что может быть лучше? Мы ходили в горы, ездили на экскурсии по древним городам, любовались на толстых нильских крокодилов. Тем более, эта поездка~--- моя заслуга! Комитет всероссийской олимпиады школьников по астрономии состоял из добрых людей, готовых оплатить победителю месячный тур в любую тёплую страну за казённый счёт. Не радоваться такой удаче~--- преступление. И всё же, дни бежали очень быстро, и каждый неумолимо приближал меня к концу лета.

Солнце всё никак не касается горизонта. И только я успеваю поверить, что день не закончится, что завтра не наступит, как пляж погружается в тень. Солнце скрылось за зеркальной морской гладью. С улыбкой на лице я иду к отелю.

\begin{center}
    * * * 
\end{center}

Я в комнате отельного номера. В шкафах царит пустота, на полу нет ничего, кроме чемоданов, стоящих аккуратным рядком. Кухня убрана, столы протёрты, на диванах и креслах блестят новые чехлы. Лишь простыня на кровати слегка помята~--- там сидит женщина средних лет в походной одежде. Я одет так же, но за моей спиной висит рюкзак. Я стою, она сидит; мы оба смотрим на утреннее солнце за широким окном. Ярко-жёлтый диск только начинает свой путь по небосводу. Конец перешёл в начало, вечернее солнце обернулось утренним. Но мы смотрим на него с печалью, потому что понимаем: оно нам больше не светит.\\

<<\dots Инвертируйте восприятие п.-в. плоскости; запускайте ускорение\dots>>\\

Вдруг женщина резко оборачивается на меня, встаёт с кровати, и реальность начинает видоизменяться. Пространство приходит в движение, выворачивается наизнанку, растягивается. Из-за дверей шкафа вылезают кленовые ветви, в щелях паркета показываются жёлтые травинки. Ножки стульев и диванов пускают корни в пол комнаты, нет, в сырую почву, покрытую инеем. Сносит крышу, в прямом и переносном смысле. Черты лица женщины стремительно меняются, печаль в глазах оборачивается усталостью и раздражением. Порыв ветра осыпает женщину осенними листьями, облачая её в новый наряд. Момент, в который трансформация пространства завершилась, я не ухватил, но сейчас передо мной широкая аллея, окаймлённая с двух сторон могучими клёнами, покрытая листьями разных оттенков красного и жёлтого. Холодный ветер пробирает меня до костей. Напротив стоит молодая девушка, лет двадцати, одетая в джинсы с модным красным ремнём и простую футболку. На голове у неё большая ярко-красная шляпа, в цвет осеннего листопада. У девушки прекрасное, но усталое лицо.

% <<\textit{Будущее}..?>>

--- Тебе не холодно?

--- Прости,~--- говорит она,~--- но я думаю, что нам нужно расстаться. Мы с тобой вместе уже четыре года, и за это время ты сильно изменился. Да и я тоже изменилась, что уж скрывать\dots \ Всё, что в тебе было,~--- это страсть, но она исчезает из отношений, и в тебе больше ничего не осталось. Ты неблагодарен, ты пользуешься правами, забыв об обязанностях, ты пытаешься усидеть на двух стульях. Прости, но мне это надоело. Ты не моешь за собой посуду, хотя я тебя просила, ты врёшь про задержки на работе\dots \ Ты постоянно врёшь\dots \ Помнишь, ты обещал прочитать рассказ, который я посоветовала тебе два месяца назад? Ты его так и не прочитал! Неужели сложно потратить двадцать минут своего времени?! Неужели обещания для тебя менее важны, чем какие-то жалкие двадцать минут?! Мне кажется, ты привык получать, не давая ничего взамен. Ты эгоистичен. И мне это надоело. Прости. И не звони мне больше, пожалуйста. Прощай.

Я стою и просто смотрю в её ядовито-зелёные глаза, не в силах произнести ни слова. Слишком многое хочется ей сказать: что она тоже изменилась, что она совсем не романтична, что она преувеличивает, я обязательно прочитаю рассказ, ведь я помню про него. Сказать, что она замечательная, что мне жаль, я исправлюсь, мне лишь нужен второй шанс\dots \ что я люблю её сильнее всех на свете и не хочу её терять. Хочется послать её к чёрту с её глупыми претензиями, и пусть она сама зарабатывает себе на квартиру. Слова толпятся в моей голове, застревают в горле. Презрительно хмыкнув, девушка поворачивается ко мне спиной и начинает удаляться. А я готов провалиться сквозь землю. И я проваливаюсь.\\

<<\dots Интегрируйте восприятие пространственных измерений\dots>>\\

Я лечу сквозь почву и каменные породы, глубже и глубже, а потом вываливаюсь из атмосферы, падая в синий океан. Я опускаю лицо в воду и вижу там глубокий небосвод, будто верх и низ поменялись местами. Или не поменялись, а стали одним целым. Небо и вода, воздух и океан~--- две стороны одной медали. Горят далёкие звёзды, призраки прошлого, пророки будущего. Я ухожу с головой под воду, поднимаюсь всё выше над небом, выходя за пределы стратосферы.

Молчаливая пустота. Я удаляюсь от Земли, прекрасной голубой планеты, в открытый космос. На мне нет скафандра, до смерти остаются считанные секунды. Я смотрю на свои руки: на одной выгравировано <<$\leftarrow$>>, на другой~--- <<$\rightarrow$>>. Или же <<$\uparrow$>> и <<$\downarrow$>>, смотря как повернуть. Звенящая тишина раздирает уши. Холодные, чужие звёзды сверлят меня безразличным взглядом. Мой же прикован к Земле, родной, милой, прекрасной колыбели Человечества. Но что я вижу? Планета продолжает жить, в то время как я начинаю задыхаться и замерзать. Голубой, с зеленоватым оттенком шар меня не видит, он меня уже забыл...

<<Посмотри на меня!>>

<<Услышь меня!>>

<<Вспомни меня!>>

Вакуум глушит крики, стирает память. Земля постепенно приобретает ядовито-зелёный цвет.

<<Таково моё будущее..?>>

Холод подбирается к моему сердцу, но не успевает на мгновение. В сердце загорается маленький фитиль, и через момент я вспыхиваю сверхновой звездой, сжигая Землю, Солнечную систему и далёкие звёзды, погружая Вселенную в небытие.\\

<<\dots Провал\dots>>

\begin{center}
    * * *
\end{center}

Сегодня я не выспался. Я не запоминаю свои сны, но в этот раз мне точно приснился кошмар, ведь встал я на час раньше времени. Чемоданы собраны, документы собраны, а родители ещё спят. Я выхожу на балкон встречать рассвет. Солнце мало-помалу показывается из-за гор. Иронично, не правда ли? Попрощавшись с ним вчера вечером, я снова вижу его перерождение, и уже через час не встречу никогда, это южное солнце. Как если бы я расстался с девушкой, и на следующий день она пришла, чтобы приготовить мне завтрак~--- и исчезнуть из моей жизни навсегда. Это если бы у меня была девушка, конечно. Сравнение смешное, но на глаза наворачиваются слёзы. Кстати, о завтраке\dots

Мы едем на такси в аэропорт. Дорога занимает не больше получаса, но ощущается в два-три раза длиннее. Я стараюсь не смотреть в окно.

В аэропорту для многих время течёт быстрее обычного. Все куда-то бегут, опаздывают на рейсы, самолёты прилетают и улетают, через динамики что-то постоянно бормочут на английском. А я сижу в зале ожидания. Телефон я забыл зарядить, и мама дала мне книжку\dots \ Видимо, чтобы время шло ещё дольше. До самолёта ещё целых три часа. Хотя бы окон в зале нет, и на том спасибо. Через динамики что-то бормочут\dots

<<There is a bomb in the building of the airport. Please remain calm and evacuate immediately. This is \textbf{not} a training. There is a bomb\dots>>

Я поднимаю глаза на родителей. Недоумение и растерянность читаются на их лицах. Спустя десять секунд мы судорожно хватаем документы и мчимся к выходу из аэропорта.

--- Мама, а чемоданы?!

--- Не отставай, Костя! Какие, к чёрту, чемоданы?!

Адреналин ударяет в кровь, сердце бешено стучит.

<<Наконец-то, время побежало быстрее>>~--- ловлю я себя на мысли. К счастью, выход оказался близко, и мы, отбежав на безопасное расстояние, сели на автобусной остановке. Папа немедленно начинает кому-то звонить, мама быстро проверяет все документы, а потом тоже достаёт телефон. Я смотрю на аэропорт, из которого выбегают люди, как крысы с тонущего корабля.

<<Взорвётся? Или нет?>>

Мой взгляд прикован к зданию. Вокруг тихо, только родители что-то энергично спрашивают и отвечают по телефону, и биение моего сердца отдаётся в ушах. Как ни странно, мне не страшно. Напротив, во мне просыпается надежда.

<<Что, если аэропорт взорвётся? Мы же никуда на полетим? Значит, мы останемся здесь, по крайней мере на пару дней? Значит, мы вернёмся в отель..? Ну давай, взрывайся!>>

Помню, где-то я прочитал мудрую мысль: <<Если ты террорист, не говори, что ты террорист>>. К счастью, мне и не пришлось.

--- Слава богу!~--- выдыхает мама, убирая телефон в карман.~--- Бомбу обезвредили, хотя виновных пока не поймали. Нам придётся подождать, пока они не проведут полный обыск здания, но взрыва уже не ожидается.

--- Это хорошо,~--- отвечаю я, но энтузиазма в моём голосе мало,~--- и мы что, даже успеем на самолёт?

--- Можем успеть!~--- весело подбадривает меня отец.~--- Главное, чемоданы нужно не забыть\dots

<<Хорошо, что взрыва не было,~--- думаю я, сидя в салоне самолёта,~--- хорошо, что никто не погиб. Хорошо.>>

Если ты террорист, не думай, что ты террорист.

\begin{center}
    * * *
\end{center}

Я стою на небе, среди белых облаков. Внизу мелькает Земля, покрытая лесами и плантациями. Надо мной белое солнце. Вокруг меня необъятные просторы, пустота на тысячи километров. Но мне всё равно тесно. Представьте себе прозрачную сферу, внутри которой находится человек. Чем ближе человек подходит к границе сферы, тем сильнее сжимается пространство вокруг него, и тем медленнее он идёт. Таков был мысленный эксперимент Анри Пуанкаре, великого французского математика. Сможет ли человек дойти до границы? Нет. Он будет видеть внешний мир, но никогда не сможет к нему приблизиться. Так и я: сколько бы ни шёл, я заперт в рамках обзора иллюминатора.

Там, далеко, за пределами моей невидимой тюрьмы, плавают дельфины, выныривая из морской пены облаков. Они двигаются с постоянной скоростью, но при этом то замедляются, то шустро ныряют в белую пучину.

<<Им тоже тесно,~--- думаю я,~--- только мне тесно в пространстве, а им~--- \textit{во времени}>>

На горизонте медленно, словно неохотно вырисовывается тёмный женский силуэт. Ни лица, ни одежды не разобрать, только большая красная шляпа угадывается на голове незнакомки, контрастируя с чёрным одеянием. Я делаю шаг в её сторону, потом другой, третий, и вот я уже бегу со всех ног к очертанию на горизонте. Но чем быстрее я бегу, тем сильнее сжимается пространство вокруг меня, а с ним растягивается время, и дельфины всё реже выныривают из облаков. Расстояние до тёмной фигуры не сокращается.

Силуэт стоит вне прозрачной сферы, за рамками иллюминатора. Его мир~--- \textit{будущее}, а я заперт в \textit{настоящем}. Как ни крути, все мы заперты в настоящем. И всё же, нестерпимое желание разбить невидимый хрусталь сферы накапливается у меня в груди, оно готово взорваться. Ещё шаг, ещё один\dots\\

<<\dots Быстрее! Оборвите проекцию временных искажений\dots>>\\

Фигура стирается бесследно.

<<И вообще, разве там что-то было?>>

Одни дельфины обитают в белом море облаков, и я стою в центре прозрачной сферы.

Мгновение, и небо уходит из-под ног. Дельфины молниеносно прорезают пространство, а я падаю на землю, набирая терминальную скорость. И превышаю её, ведь время терминальной скорости не имеет.

Подо мной здание каирского аэропорта.

Страшно ли мне? Нет. Адреналин ударил в кровь, сердце бешено стучит, но мне не страшно. Напротив, огонь надежды светится в моих глазах. Огонь спички, поджигающей фитиль. Ветер свистит в ушах, заглушая мысли.

% <<\dots Когнитивный регресс необратим\dots>>\\

Столкновение. Аэропорт взрывается.\\

<<\dots Провал\dots>>

\begin{center}
    * * *
\end{center}

--- Ты все книжки на лето прочитал?

Мама умеет нас\'{ы}пать соль на рану. Вчера вечером мы прилетели в Петербург, а завтра, аккурат в понедельник, мне идти в школу.

--- Жизнь~--- боль\dots \ Такое ощущение, что сентябрь начинается в августе, а понедельник~--- в воскресенье.

--- Не так, Костя. Понедельник начинается в субботу. Это, между прочим, тоже было в списке литературы.

Сегодня утром я проснулся с хорошим настроением. <<Перед смертью не надышишься>>,~--- шутят родители. А я отвечаю, что им самим завтра выходить на работу, и улыбка быстро сходит с их лиц. Я никогда не занимался танцами, но свою последнюю попытку насладиться каникулами, свой лебединый танец я исполняю грациозно. Учебники из школы забрал, в парк аттракционов с друзьями сходил, яблок по скидке купил, на велосипеде покатался, и глядишь~--- уже вечер. Я словно террорист перед публичной казнью, который смело смотрит в глаза палачу и весело кричит: <<Ну давай, не тяни! Я не кусаюсь!>>, доводя палача до бешенства.

У меня странная школа. Если обычно в средних классах школьники имеют право самовыражаться через одежду, в рамках приличия, естественно, то у нас введена строгая форма: кеды, брюки со стрелками, рубашка, пиджак и жилетка. И всё смольно-чёрное. Дресс-код един для учеников и учителей, так что всем приходится жариться под слоями шерстяной ткани. Всем, кроме директора школы. Она всегда носит чёрное, как ночь, бархатное платье и ярко-красную шляпу. Для своей профессии она чрезвычайно молода~--- считается, что ей не больше тридцати лет, хотя доподлинно никто не знает. <<Наше дело~--- будущее, к чему разговоры о прошлом>>,~--- говорит обычно она, когда кто-то интересуется деталями её карьеры, и скрывает лицо под широкой шляпой. Не мудрено, что про неё ходит множество слухов: что она секретный агент под прикрытием, или пришла из будущего, или же что она вообще не человек и живёт \textit{вне времени}. Все они, однако, остаются лишь догадками или страшными историями у костра.

История моего поступления в частную общеобразовательную школу <<Грядущее>> тоже довольно занимательна. Два года назад мы переехали в центр города, и, буквально через день после переезда, с моими родителями связалась лично директор школы, Ариана Артуровна Цукунфт-Ферганген, с предложением принять меня в своё новое заведение.

--- Ну и имечко...~--- изумлялась моя мама, читая электронное письмо.~--- Иностранка из Германии, наверное? Зато школа у неё, я смотрю, солидная. Всего год назад открылась, а сколько положительных отзывов! И место в рейтинге у них приличное\dots \ Костя, можешь начинать готовиться к вступительным экзаменам.

\begin{center}
    * * *
\end{center}

Я сидел на деревянной пристани внутри безлюдного залива, свесив ноги с пирса и почти касаясь пальцами прозрачной воды. Позади, на берегу, робко шелестели деревья, тёплым и мягким светом солнце покрывало побережье. Я слушал ровный шум прибоя и глядел в небо. Лёгкие перистые облака смущали его лазурную пастельную голубизну, плавно переходящую в оранжевую, а затем и кроваво-красную закатную акварель. 

<<Небо~--- одно из тех поразительных явлений, которые неизменно сопровождают нас всю нашу жизнь. Время им нипочём. Такие природные явления, будь то заснеженные горные вершины, непроглядные лесные чащи или далёкие звёзды, неизбежно порождают в людях яркие чувства, иногда радостные, но чаще печальные, и напоминают о неотвратимости смерти. О том, что будущее когда-то станет настоящим, которое тут же становится прошлым. Эти три временные точки отождествляются при взгляде в глубокий океан неба, который помнит и прошлое, и настоящее, и будущее.>>

Я смотрел на свои тёплые воспоминания. Смотрел на друга, с которым прошло всё моё детство, и испытывал чувство горькой ностальгии по времени, проведённому вместе с ним, погружался в \textit{прошлое}. Август, разбиватель сердец.

Розовое солнце касалось морского горизонта, его отражение танцевало лебединый танец на волнах.

Со стороны леса выползали смольно-чёрные грозовые тучи. Они терпеливо подкрадывались к солнцу. А я смотрел на его нежные лучи через линзы слёз, пытаясь выгравировать в памяти каждый оттенок вечернего небосвода.

Всё когда-то заканчивается. Вода пошла рябью, а тучи стали поглощать лучи солнца один за одним, пока на залив не упала тень. Слёзы замёрзли на моём лице.

За моей спиной стояла женщина в чёрном, как ночь, бархатном платье и большой красной шляпе. Головной убор был таким ярким в тени залива, словно сам излучал еле заметное красное свечение. Женщина тихо подошла ко мне, а затем обняла меня со спины.

--- Ни к чему сокрушаться о прошлом,~--- мягко произнесла она,~--- наше дело~--- будущее.

Я поднял глаза на смольно-чёрное грозовое небо. Среди туч блестели красные, синие, фиолетовые молнии. Каждая из них обладала своим неповторимым оттенком, а вместе они складывались в загадочную симфонию, рисуя картины \textit{будущего}. Страшные, но интригующие картины.

\begin{center}
    * * *
\end{center}

<<\dots Стимулируйте отделы мозга, относящиеся к первому эпизоду сновидения\dots>>\\

Пляж. Ноги утопают в море жёлтого песка, глаза устремлены в бескрайний синий океан. Кругом пустота. Я иду по береговой линии, приливные волны равномерно омывают мои стопы. Солнце над головой горит ядовито-зелёным светом. Я иду и смотрю себе под ноги. Зелёный свет отражается от каждой песчинки, проходя сквозь водную призму и создавая на песчаном дне причудливые узоры. Волны накатывают одна за другой, отталкивая, сталкивая песчинки между собой и снова утаскивая их в морское лоно. Воздух наполнен шумом прибоя, шипит, разбиваясь о песок, морская пена.

<<Разве это пустота?>>

Только данная мысль звучит в моей голове, все звуки исчезают. Каждая песчинка, каждая капля морской воды оборачивается холодной гравитационной сингулярностью, жадно хватает ядовито-зелёные фотоны света. Пространство-время сжимается в точку, и меня засасывает в \textit{Пустоту}. Если весь мир~--- это совокупность информации, полученной органами чувств человека, то никто никогда не видел, не слышал, не чувствовал \textit{Пустоту}. Зрение, слух, обоняние, осязание, вкус и чувство равновесия молчат; эмоции мертвы. Ничего нет, кроме безмолвной \textit{Пустоты}. И меня тоже нет.\\

<<\dots Провал\dots>>

\chapter{Сентябрь}

\epigraph{\textit{Время всегда сдерживает обещания. Если август~--- это начало, то сентябрь~--- логический конец, обещанный удар судьбы. Ложась под гильотину, в последние минуты жизни человек явно представляет, как нож со скрипом падает на его шею. Но разве, прокручивая неминуемую смерть в голове раз за разом, он не надеется в глубине души, что палач промахнётся топором? Что старая верёвка завяжется в узел? Что ржавое лезвие застрянет в брусьях? Конечно, надеется. Однако опытный палач бьёт верно, верёвка скользит по дереву, и отполированный нож со свистом летит вниз. Надежда умирает последней, и сентябрь ответственно провожает её в загробный мир.}}

\textit{Он} шёл по безымянной улице Города Мёртвых. Его окружали серые каменные дома с облезлыми черепичными крышами, за которыми скромно прятались глубокие выгребные ямы. Из окон то и дело выливались нечистоты, скапливаясь на обочине, и плыли медленным потоком вниз по улице.

Неба не было. Ни намёка. Крыши домов уходили в густой серый туман, а освещение давали факелы и фонари. Без факела, в принципе, в городе было никак~--- даже у земли туман давал видимость не больше чем на десять метров. Холодный, зыбкий, удушающий туман. Говорили, что, попав в неосвещённый район города, можно было поплатиться жизнью. Впрочем, жизнью тут все уже поплатились. Заходя в туман, люди теряли куда больше, чем жизнь.

Где начинался и заканчивался \textit{Аид}, никто не знал. Считалось, что его центром были огромные Нефритовые Врата, украшенные, скорее изуродованные, сотнями каменных скульптур. Шесть водных каналов окружали эти врата: Стикс, Ахарон, Лета, Флегетон, Коцит и Океан. Пересекая каждый из них, человек лишался одного из шести чувств: зрения, слуха, осязания, обоняния, вкуса и равновесия. Говорили также, что Стикс неестественно светился, Ахарон оглушительно шумел, а Флегетон издавал зловонный запах. Где причина, а где следствие~--- неизвестно.

Он шёл по главной улице города. Несмотря на средневековую европейскую планировку, вывески встречали его надписями на древнегреческом. Идти было трудно~--- дорога кишела людьми. Сгорбленные, иссохшие, в оборванных тряпках, некоторых сложно было счесть за людей. Несмотря на давку, город пребывал в мёртвой тишине.\\

<<\dots Ну и дыра\dots>>\\

Душная атмосфера безысходности давила серым туманом. Аид не знал компромиссов, никогда не давал второй шанс, а иногда не давал и первый.

Он шёл по главной улице на площадь. Шаг, другой, третий\dots \ Он не думал, он шёл. Шаг, другой, третий\dots \ Люди провожали его не то голодными, не то испуганными взглязами пустых глазниц. Шаг, другой, третий\dots\\

На пустой площади стоял высокий обсидиановый обелиск. А перед ним~--- женщина в чёрном, как ночь, похоронном убранстве и красной шляпе. Шляпа заметно тускнела и лишь чудом сохраняла свой цвет. Миг, и шляпа стала серой. Чудес не бывает.

--- Врата, --- монотонно говорит женщина, --- твой путь лежит к Нефритовым Вратам. Иди по главной улице, и не ошибёшься. Тебе придётся пересечь шесть каналов, выпав в \textit{Пустоту}, но если тебе хватит воли, то ты найдёшь выход. По легендам, пройдя сквозь Нефритовые Врата, можно вернуться в мир живых, некую \textit{реальность}.

% <<\dots Проанализируйте\dots \ Что значит не поддаётся анализу?..>>

\begin{center}
    * * *
\end{center}

На календаре середина сентября. В лучших традициях Петербурга, небо заволокло тучами, а на землю спустился туман. Опять заводы, скорее всего, портят воздух\dots

Время всегда сдерживает обещания. А иногда оно превосходит все возможные ожидания.

Мама погибла 1 сентября, 15 дней назад. Как? Очень просто, в автокатастрофе. По словам свидетелей, какой-то лихач поехал на красный, а мама переходила в этот момент дорогу. Лихач после этого как будто испарился.

В то утро я вышел в школу, полный уверенности, смирившись со своей горькой судьбой. Какой? Горькой? \textit{Горькой?} Вернувшись домой, я потерял одного родителя из двух. В тот день туман упал на город.

--- Папа, скажи, а разве люди умирают просто так, случайно? Без всякой причины? Раз, и нет\dots \ --- спросил я тогда своего отца.

--- Да, Костя, умирают, --- ответил он, давясь слезами и безуспешно стараясь это скрыть, --- ты, главное, ходи в школу, ладно? Ладно? Что бы ни случилось, ходи в школу. Я со всем разберусь, ты только не бойся. Всё будет в порядке, да?

--- Хорошо, папа, я буду ходить в школу. Да.

С чем именно папа хотел разобраться, я так и не понял. Днём он сидит на кухне в окружении банок из-под пива, а ночью рыдает в подушку. Я никогда не видел его таким.

А я ведь плакал \textit{тогда}. Когда провожал закатное солнце, сидя \textit{на пристани в заливе}.\\

<<\dots Фиксируйте\dots>>\\

Тогда ледяные слёзы градом катились по моему лицу. Почему же сейчас, когда нужно плакать по-настоящему, я не могу выдавить из себя ни капли?! <<Август, разбиватель сердец>>? С кем там я себя сравнивал? Террорист перед публичной казнью? Чёрта с два! Пока я тешился дешёвой лирикой, пока лил слёзы по пустякам, случился настоящий кошмар.

<<Перед смертью не надышишься!>>

Да уж, надышался!

<<Ты все книжки на лето прочитал?>>

Понедельник начинается в субботу? В воскресенье? Нет, чёрт возьми, понедельник начинается в понедельник!

Как и обещал, я хожу в школу. Встал, оделся, не забыл поесть, дальше первый урок, второй, третий\dots \ Вызывают к доске. Выхожу, отвечаю, сажусь. Учителя стесняются со мной говорить, одноклассники смотрят с сожалением, как на смертельно больного или на дауна. Иду домой, делаю домашнее задание. Потом просто лежу в кровати до следующего утра. А, и ещё поесть. Три раза в день.

А в голове постоянно один вопрос: зачем? Нет, не так: Зачем?! Зачем куда-то идти, зачем что-то делать, если люди умирают без всякой причины? Зачем маму убили? Что она сделала?

<<Зачем? А действительно, зачем маму убили? Должна же быть причина! У полиции нет зацепок, ну и что? Я уж точно что-нибудь найду>>.

Завтра будет мой последний день в школе.

\begin{center}
    * * *
\end{center}

Он медленно тонул в чумном водоёме Аида. Этот водоём назывался Озером Разложения, и на то были причины. В озеро стекались нечистоты со всего города. Мутная, вязкая вода кишела болезнетворными бактериями, червями и личинками. Некоторые считали, что на дне Озера Разложения находился проход в Тартар, откуда можно было попасть напрямую к Нефритовым Вратам. Этот путь, однако, был гораздо мучительнее пересечения шести водных каналов.

Он медленно погружался на глубину. Под ним в мутной воде постепенно проглядывался странный рельеф. Острый каменный выступ. Ещё один, а вот и третий. <<Треугольники>> скал образовали последовательность, закручивались в спираль.

<<Зубы>>, --- подумал он.

Из глубины на него смотрел жёлтый глаз с зеленоватым зрачком, около метра в диаметре. Рядом с глазом на поверхности \textit{существа} множились капсулы с жиром и гноем. Они периодически лопались, исторгая в воду тошнотворную субстанцию. Он подплывал ко рту \textit{существа}, который всё больше сужался и расширялся, словно в мерзком предвкушении добычи.

Вокруг зубов плавали русалы. Они попеременно подплывали к нему, заглядывали в лицо. Изуродованные какими-то ритуалами рыбьи хвосты, гнойные, заплывшие жиром тела, лысые головы. Но страшнее всего были их лица. Рот неестественно широкий, с тонкими, еле заметными губами и острыми зубами. Две чёрных дырки вместо носа. И глаза без век, в которых не было ничего человеческого. Эти глаза смотрели ему прямо в душу, сводили его с ума.

Черви ели его глаза, пиявки забирались ему в уши, пока он тихо опускался в пасть \textit{существа}. Русалы провожали его голодными взглядами, но им было нельзя.\\

<<\dots Можешь подежурить за меня? Мне надо сходить проблеваться\dots>>

\begin{center}
    * * *
\end{center}

Я сижу на уроке биологии. За окном серый туман. Больше, чем вчера.

--- \dots Откройте учебник на странице\dots

--- Коршунов!

Я открываю рот, чтобы сказать, что не сделал домашнее задание, но в последний момент понимаю, что голос шёл из-за спины. В дверях класса стоит Ариана Артуровна, директор школы <<Грядущее>>. Все в классе удивлены: она редко выходит из своего кабинета.

--- Константин Коршунов, можно вас на минуту?

Ничего не понимая, я встаю и молча следую за дамой в ярко-красной шляпе и чёрном, как ночь, платье.

--- Простите, а куда мы идём?

--- В мой кабинет, Костя. У меня к тебе важный разговор.

<<Важный разговор? Меня что, исключат? Из-за того, что мы не можем оплачивать обучение?>>

Кабинет директора отличается от всех, что я видел. За красной дверью открывается белая, идеально сферическая комната. Из пола вырастает белый стол в форме большой фасоли, на нём стоят компьютер и принтер, а также лежит различная канцелярия. За столом на город выходит широкое овальное окно. От кабинета веет футуризмом. Ариана Артуровна садится на белый дизайновый стул, предлагая мне сесть на второй.

--- Так в чём же дело? --- начинаю я с лёгким раздражением. --- Вы меня исключаете?

--- Ничего подобного,~--- с тенью усмешки отвечает она,~--- ты, конечно, человек исключительный, но исключать тебя не за что. Костя, ты заметил, что туман в городе необычно густой? Это, конечно, Петербург, но не так же\dots \ Более того, некоторые граждане жалуются, что слышат некие голоса сверху, или даже видят огромные торчащие зубы. Это же\dots \ \textit{твоя работа}?

--- Что? Я вас вообще не понимаю,~--- раздражение в моём голосе нарастает,~--- почему вы говорите загадками? Зачем вы меня сюда позвали?

--- Костя, я понимаю, тебе тяжело,~--- она посмотрела на меня своими ядовито-зелёными глазами,~--- всегда тяжело терять любимых людей. Но тебе не стоит зацикливаться на прошлом. Я покажу тебе \textit{будущее}.

--- Да что вы заладили со своим будущим?! Что за бред вы несёте?!~--- я взрываюсь.~--- Что вы вообще знаете?! Кому нужно такое будущее?! Да я бы жизнь десять раз отдал за возможность вернуться в прошлое! Спасти маму!\\

<<\dots Наблюдаются скачк\'{и} онейронной активности\dots>>\\

Только я собираюсь убраться к чёртовой матери из этого проклятого футуристичного кабинета, за окном мелькает что-то, похожее на рыбий хвост. А в следующий миг в меня впивается пара голодных, неморгающих глаз. Неестественно широкий рот складывается в зловещую улыбку.

<<\dots Небо и вода --- две стороны одной медали\dots>>\\

<<\dots Успех. Фиксируйте\dots>>\\

--- Костя! Нам нужно уходить, --- Ариана Артуровна берет меня за руку, --- быстрее!

Я молча следую за директором. Меня терзает чувство d\`{e}j\'{a} vu. Но где я мог видеть такую тварь? Эти холодные, нечеловеческие глаза без век? Неужели они мне\dots \ \textit{снились}?

Директор на бегу достаёт телефон, быстро отдаёт приказы.

--- Костя, не отставай! Эвакуация уже пошла. Но мы с тобой пойдём в другое место\dots

Мы забегаем в библиотеку~--- самое новое помещение в школе. Только в прошлом месяце пристроили. На каменной стене за стойкой библиотекаря Ариана Артуровна что-то нажимает. Стена раздвигается, открывая перед нами винтовую лестницу, ведущую вниз.

--- Скорее! --- директор увлекает меня за собой. Она щёлкает пальцами, и перед нами возникает яркий огонёк. Он освещает нам путь. Сверху доносится приглушённый гул. Моё воображение уже рисует в конце лестницы голодного русала с широкой улыбкой, но спираль приводит нас во вполне уютную комнату, обставленную на старомодный лад. Камин, ковёр, диван, кресла-качалки с пледами. Директор запирает дверь на три больших засова, потом с облегчением выдыхает и сваливается на кресло.

--- Ариана Артуровна, что это вообще было?~--- наконец, спрашиваю я.~--- Откуда здесь монстры из моих снов? Это были вещие сны, или что-то вроде того?

--- Если вкратце, --- начинает директор, --- ты \textit{воплотил} этих русалов из своих снов. И сны эти были никакие не вещие. Не реальность повлияла на сновидения, а наоборот, понимаешь?

--- Ничего не понимаю\dots \ Магия, что ли? И к тому же, как вы создали тот огонёк на лестнице?

--- Костя, я отвечу на все твои вопросы. Но перед этим тебе нужно ненадолго заснуть. Я покажу тебе своё прошлое~--- и твоё будущее.

Ариана Артуровна встаёт, указывает рукой на диван. Мол, <<ложись>>.

--- Никуда я не буду ложиться! --- я отступаю назад. --- Разве можно сейчас спать? И вообще, что вы со мной собрались делать? Вы, случаем, не один из моих кошмаров, заодно с этими тварями?!

--- Костя, ты умный парень. Но если ты хочешь получить ответы на свои вопросы, тебе остаётся только довериться мне. Прошу тебя, умоляю\dots \ Ты единственный, кто в силах спасти город, а может, и весь мир от собственного кошмара.

Не хочется признавать, но вариантов у меня действительно мало. Ну что же, как говорится, <<делай что должен, и будь что будет>>. Я ложусь на мягкий диван, закрываю глаза.

--- Спасибо, Костя, --- говорит директор. Она касается пальцами моих висков, и я проваливаюсь в сон.

\begin{center}
    * * *
\end{center}

Её первым воспоминанием было солнечное утро в маленькой уютной комнате. У стен аккуратные шкафчики, на полу розовый узорный ковёр, а через широкие прямоугольные окна девочку заливал тёплый солнечный свет. Девочка не знала ни своего имени, ни возраста.

Позже ей объяснили, что ей пять лет, и что зовут её Ариана Артуровна Цукунфт-Ферганген. А находилась она в доме для особенных детей имени некоего Константина Коршунова. Кроме того, ей объяснили самое главное. Ариана жила \textit{назад}. То, что для всех было прошлым, для неё было будущим, и наоборот. Девочка не сразу поняла, но потом разобралась. Каждый раз, просыпаясь утром, она оказывалась днём \textit{ранее}. После осени для неё наступало лето, потом весна и зима. Весь мир уже знал Ариану Цукунфт-Ферганген, не знала только она сама. А на следующий день она исчезла, как будто её никогда и не было. Но то было на \textit{следующий} день, а Ариана, заснув под вечер, попала в \textit{предыдущий}.

Девочка проводила дни за играми и чтением книжек. Иногда она выбегала во двор и играла с другими детьми, но дружб не заводила, потому что на следующий день дети её забывали, а некоторые винили Ариану в том, что та забыла их. К девочке приходили незнакомые люди, они рассказывали о себе. И все они пытались заставить Ариану что-то вспомнить. От этого Ариане было грустно; она чувствовала себя виноватой.

Чаще всех приходил стройный мужчина лет сорока, в широких штанах и коричневом пиджаке на простую футболку. У мужчины были умные, добрые глаза.

--- Как вас зовут? --- спросила Ариана, когда мужчина пришёл в первый раз.

--- Костя.

--- А всё имя? --- настаивала девочка.

--- Константин Андреевич Коршунов, к вашим услугам, --- с нарочитой торжественностью произнёс Костя.

--- Здразввтуйте\footnote{Не опечатка, а имитация детской речи!}, Константин, а меня зовут Ариана Цук\dots \ Цукун\dots \ В общем, Ариана.

--- Приятно познакомиться, Ариана.

Костя был единственным, кто не пытался пробудить в девочке давние, ещё не полученные воспоминания. Хотя в нём этих воспоминаний хранилось больше всего.

Годы шли, Ариана становилась старше, а мир вокруг становился моложе. Костя приходил каждый день, и они вместе читали книжки, учились писать, занимались математикой, биологией, информатикой. Каждый следующий день Костя спрашивал Ариану, чему он учил её \textit{вчера}, чтобы знать, чему учить сегодня. Они гуляли по паркам, и Костя рассказывал девочке о мире.

--- Скажи, а это в честь тебя назвали наш приют? Ты что, какая-то знаменитость?

--- Да, в честь меня\dots \ --- смущённо отвечал Костя. --- Около семнадцати лет назад я остановил крупномасштабную войну, и теперь все считают меня\dots \ как это\dots \ героем, что ли.

--- Ух ты! --- восхищалась Ариана. --- Какой же ты классный! Вот бы и мне такой стать!

--- Ну ладно, ладно\dots \ Не сотвори себе кумира. Скажи лучше, не хочешь завтра пойти на концерт?

--- На концерт? Так мы же вчера уже ходили\dots\\

Ариане было десять лет. Они  сидели в её комнате.

--- Скажи, Ари, тебе часто снятся сны?

--- Часто. А что?

--- Что тебе последнее снилось?

--- Мне снилось деревце, которое растёт из горшка. Все деревья вокруг уменьшаются, а я хочу, чтобы они росли\dots

--- Очень хорошо, --- Костя достал из-за спины горшок с землёй (где он его прятал?), --- положи руки на горшок и закрой глаза. Да, вот так. Теперь вспомни свой сон так чётко, как только можешь, представь, как твоё деревце растёт из горшка. Хорошо, открывай глаза.

Из горшка торчал тонкий ствол, расходившийся ещё более тонкими ветками с робкими зелёными листиками.

--- Костя, что это такое?! Как так получилось?

--- Запомни, Ари. Сны --- это обратная сторона нашего мира. Ты же замечала, что во снах часто повторяются события из твоей жизни? Так почему же нельзя повторить событие из сна в реальности? Такой приём называется \textit{обратным воплощением}. У него множество применений. Ты не против, если я тебе кое-что \textit{покажу}?

Ариана кивнула. Костя коснулся указательными пальцами её висков, и она провалилась в сон.

\begin{center}
    * * *
\end{center}

Ариана пар\'{и}ла в небе, пролетала сквозь облака и проносилась над лесами и озёрами.

\begin{center}
    * * *
\end{center}

--- Ты же знаешь, что делать?

Девочка сосредоточилась, вспомнила ощущение полёта, буйство эмоций, и её ноги оторвались от пола комнаты.

--- Ты большая молодец, Ари. У тебя замечательный потенциал. Знаешь, я ищу проход в \textit{реальность}. Если тебе приснится что-то подобное, пожалуйста, скажи мне, хорошо?

--- Какую такую реальность? А здесь, что ли, не реальность?

--- Я и сам не знаю, Ари. Просто пообещай мне, что расскажешь, если тебе приснится \textit{реальность}.

--- Хорошо.\\

Прошло ещё пять лет. Ариана освоила базовую школьную программу и сдала государственные экзамены. Теперь она могла посвящать больше свободного времени \textit{обратному воплощению}. Эта способность приводила девушку в восторг, погружала в себя без остатка. Ариана научилась \textit{транслировать сновидения} --- \textit{воплощать} их в сознаниях других людей.

Мир становился всё моложе и моложе. Деревья постепенно уменьшались, дома становились новее. Литература исчезала из публикаций, архитектурные стили шагали назад во времени. Модели айфонов шли в обратном порядке.

Стояла ранняя осень. Они шли по широкой кленовой аллее, ворох красных и золотых листьев покрывал землю и пронизывал воздух. На Ариане были джинсы с модным красным ремнём и простая футболка, а на голове сидела ярко-красная широкополая шляпа, которую Костя когда-то подарил ей.

--- Я люблю тебя, --- бросила Ариана, и тут же покраснела до ушей. Разве может 15-летняя девушка сказать такое 30-летнему мужику?! --- В-в смысле, не в этом смысле\dots

--- Ха-ха, я тоже тебя люблю, Ари, --- рассмеялся Костя. Ариана зарделась ещё сильнее.\\

В свои двадцать лет Ариана Артуровна вступила на пост директора школы <<Грядущее>>. Точнее, она обнаружила, что вчера была на пенсии, а сегодня уже (ещё?) нет. По-видимому, школу в будущем основала сама Ариана, и назвала её соответствующе. Работа не требовала много времени, и девушка посвятила себя серьёзной науке --- исследовательской онейрологии. Какие сны бывают? Какие сны легче или труднее \textit{воплотить?} От чего зависит \textit{воплощение} и на что влияет? На все эти вопросы Ариана пыталась найти ответы.

Кроме того, между ней и Костей завязались романтические отношения. Ариана сама толком не помнила, как именно они завязались, но её всё устраивало. Ничего удивительного в таком положении дел тоже не было: оба были молоды и привлекательны, к тому же, давно знали друг друга. \\

%Их совместная жизнь выглядела немного странно:

% --- Мы вчера ходили на свидание?

% --- Да. Мне очень понравилось.

% --- Хорошо, жду не дождусь!\\

В свои двадцать лет Константин Коршунов ушёл на войну.

--- Ари, я же вернулся к тебе вчера?

--- Да, Костя, ты вернулся. Но ты не хуже меня понимаешь, что моё прошлое --- лишь версия твоего будущего! Знаешь, я боюсь. Боюсь, что ты уйдёшь, а я так и не узнаю, вернулся ли ты.

--- Нельзя бездействовать, если можешь что-то изменить. Ты сама меня так учила. По правде, я должен был действовать гораздо раньше, но раз за разом закрывал глаза на предпосылки, делал вид, что это не моё дело. Уже начинаются боевые действия, понимаешь? Сомневаться поздно. К тому же, ничего они мне не сделают.

За свою не самую долгую жизнь Костя видел много снов, очень много. В одних из них он обладал телекинезом, в других --- телепатией или другими сверхестественными способностями. И теперь пришло время воплотить эти сны в реальность, насколько хватит его параметра WP. Но на следующий день для Арианы Костя уже никуда не уходил, а только собирался. А ещё через день --- даже не думал о войне.

--- Скажи, тебе не грустно жить \textit{назад?} Из светлого будущего в тёмное прошлое? Не грустно наблюдать технологический и социальный регресс?

--- Что есть прогресс, а что регресс? Правда ли, что виниловые пластинки хуже, чем электронные плееры? В чём-то да, а в чём-то --- нет. Ваше тёмное прошлое --- это моё светлое будущее. Но ты прав, мне грустно. Грустно потому, что моё прошлое --- лишь призрак твоего будущего. Может, сбудется, а может и нет. Что я тогда вообще знаю? Ни прошлого, ни будущего\dots

--- Посмотри на проблему с другой стороны. Весь мир не знает, что с ним случилось в прошлом. И только ты пишешь его историю.\\

Годы шли, и люди постепенно теряли память об Ариане Цукунфт-Ферганген. В конце концов, она стала учить Костю тому, чему тот учил её --- обратному воплощению.

--- Ариана Артуровна, вчера вы меня ничему не учили, --- ответил однажды мальчик Костя на обычный вопрос.

--- Хорошо. Скажи, тебе часто снятся сны?

\begin{center}
    * * *
\end{center}

Прошло около трёх минут.\\

<<\dots Тревога! Мы потеряли контроль над онейронной системой! Когнитивная структура перестроена\dots>>\\

--- Пойдём, Ари, пора закончить этот кошмар.

Ариана очаровательно улыбается.

--- Здравствуй, Костя!

Мы поднимаемся по винтовой лестнице, затем выходим во двор школы. Перед нами разворачивается страшная картина: всё заволокло серым, зыбким туманом, вместо неба --- необъятная гора плоти, из которой спиралями торчат зубы и таращатся глаза. На земле лежат тела людей, и над каждым по две-три фигуры. Только завидев, русалы бросают свой кровавый аперитив и разом направляются ко мне. Я поднимаю руку, и зубастые твари замирают перед невидимым барьером. Хотя, лучше сделать его видимым, так спокойнее. Я закрываю глаза. Я вытаскиваю из памяти образ августовского, ещё солнечного Петербурга.

--- Больше не будет жертв. Больше не будет кошмара.\\

<<\dots Фиксируем широкий всплеск онейронной активности\dots>>\\

Я открываю глаза. Светит солнце.

--- Ты молодец, --- Ариана обнимает меня со спины.

--- Ари, мне никогда не снился Петербург.

--- Что?

--- Я вытащил образ \textit{из памяти}. Кроме того, ты ведь тоже транслировала мне не сон, а свою \textit{память?}

--- И правда\dots \ Но как это возможно?

--- Там, в будущем, я искал проход в \textit{реальность}. К сожалению, я так и не сказал тебе, зачем. Однако теперь у меня есть догадка. А что, если мы спим? Если наша нынешняя реальность --- коллективное сновидение? Поэтому мы можем воплощать его в самом себе.

--- Весь этот мир --- сон?! Чёрт, вполне возможно\dots

--- Это не всё, Ари. Сейчас я знаю то, чего не знал в твоих воспоминаниях. Нам нужны Нефритовые Врата.\\

В \textit{Аид}, Город Мёртвых, легко попасть. Забудешь такое\dots \ Несколько секунд, и перед нами Нефритовые Врата, изуродованные сотнями каменных статуй. Я беру Ариану за руку.

--- А это точно то, что нужно? --- неуверенно спрашивает Ариана.

--- Есть только один способ узнать. Делай что должен, и будь что будет.

Мы переступаем врата.

\chapter*{Эпилог}

\dotsОн нажимает красную кнопку -19 этажа. Подземного этажа клиники, до которого не добивает сеть <<нейринов>>. В лифте Антон достаёт сводку по испытуемым. На вершине списка --- <<Константин Коршунов, WP: 7032>>.

<<Вы там что, опять свои секретные эксперименты на людях проводите?>>

Никита Высокогорный, передовой онейролог, оказался прав. <<Оптимизаторы сновидений>> не вершина прогресса, и учёный это понимал. По всему миру были разбросаны подземные лаборатории, в которых проводились секретные правительственные исследования феномена <<обратного воплощения>>. Целью учёных был контроль феномена и, возможно, создание мощного онейронного оружия --- <<ядерной бомбы>> нового поколения. Для этого изолировались сироты с высоким показателем WP, которых потом вводили в длительный сон, в \textit{псевдореальность}. Внутри неё испытуемые могли видеть свои собственные сны. Такой подход позволял учёным наблюдать за естественным <<обратным воплощением>> без угрозы реальному миру. Кроме того, можно было влиять на процесс, правда, лишь на базовом, фундаментальном уровне: ускорить, замедлить, инвертировать восприятие. Новейшие разработки также позволяли полностью устранять чётко выраженные аспекты сновидения.

Лифт приезжает. Перед старшим врачом открывается длинный коридор с дверями по бокам. Дверей, наверное, штук тридцать, не меньше.

--- Ну что, посмотрим, как там наш коршун поживает, --- Антон открывает дверь с номером <<1>>.

В блоке царит полный швах. На мониторах компьютеров быстро ползут строки, графики скачут, числа растут с невообразимой скоростью. Учёные бегают туда-сюда, пытаясь перекричать друг друга.

--- Мы потеряли контроль!!

--- Что это было?! Как он \textit{воплотил} реальность?!!

--- Идиот! Это осознанное сновидение! Стоп, что?!

--- Напряжение растёт! Сделайте что-нибудь!!

Только один человек в комнате остаётся в спокойствии. Перед рядами столов с беснующимися учёными, за толстым бронебойным стеклом, на кушетке лежит мальчик лет тринадцати. К его голове подсоединена сеть электродов, рядом стоит капельница с пищевым раствором.

--- Всем молчать! --- оглушительно рявкнул Антон, и все замолчали. --- Доложите ситуацию!

--- Так точно! Мы потеряли контроль над онейронной системой испытуемого. Мальчик понял, что находится во сне. Однако, мы совершили поразительные наблюде\dots

Учёный не успевает договорить. Стекло с треском разлетается взребезги, а мальчик поднимается на кушетке, отрывает от себя электроды и сосуд капельницы. Под изумлёнными взглядами всех присутствующих, он свешивает ноги с кушетки и облокачивается на колени.

--- Ч-что\dots

--- Где Ари? --- холодным, твёрдым голосом спрашивает Константин Коршунов. Он говорит тихо, но все его слышат. --- Где Ариана Цукунфт-Ферганген?

--- Как ты смог проснуться?! Ч-что ещё за \textit{Н-Нефритовые Врата?} --- быстро, сбивчиво спрашивает один из учёных.

--- Я не отвечу ни на один ваш вопрос, --- Костя сверлит Антона, точно, именно его, ледяным взглядом, --- пока вы не ответите на мой.

--- Арианы Артуровны Цукунфт-Ферганген никогда не существовало, --- спокойно говорит Антон, --- она --- плод твоего воображения. В возрасте пяти лет, стоя перед горящим домом своих бабушки и дедушки, ты испытал мощнейший эмоциональный всплеск. Тобой двигало простое желание --- вернуться в прошлое, жить \textit{назад}. Мы предполагаем, что результатом стало появление в твоей псевдореальности человека по имени Ариана.

--- Так вы наблюдали за мной всё это время? --- Костя на пару секунд задумывается, а потом его лицо становится каменным. --- \textit{Это же вы убили мою мать?}

Короткая пауза.

--- Нет, --- спокойно отвечает Антон.

--- Вы совсем не умеете врать, Антон Валентинович, --- атмосфера в блоке накаляется.

--- И вообще, --- попытался разрядить обстановку один из лаборантов, --- что это за <<трансляция сновидений>>?

--- Да вы, я вижу, толком и не понимаете, что такое \textit{обратное воплощение}, --- с горечью произносит Костя. --- Каждый человек имеет свой собственный мир снов, по богатству сопоставимый с объективной реальностью. Так что мешает \textit{воплотить} образ из своего мира в мир чужой? То, что вы называете <<обратным воплощением>> --- лишь прообраз, подсознательная оболочка величайшей ментальной способности человека, создающей контакт между мирами, --- напряжение в голосе Кости нарастает. --- Хотите, я прямо сейчас погружу всю эту комнату в \textit{Пустоту}? Или, может, перенести вас в \textit{Аид}? Скормить \textit{существу}? А? Что, доигрались?! И никакие \textit{нейрины} вам не помогут!

Под потолком начинает скапливаться серый туман. Мигают красным сигналы тревоги, оглушительно ревёт сирена. Наступает паника. Мальчик улыбается. Во всеобщем замешательстве Антон выхватывает пистолет из кобуры, и мальчик падает с пулей в голове.

--- Спи, коршун. Тебе не стоило просыпаться.\\

О секретном государственном проекте <<Сон наяву>> не осталось никаких записей. Антон Валентинович Страусов, главный врач 5-й Санкт-Петербургской государственной онейрологической клиники, лечит пациентов.

\end{document}
